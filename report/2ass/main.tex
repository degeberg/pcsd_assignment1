\documentclass[a4paper,final]{article}

\usepackage{a4wide}

% Language and encoding
\usepackage[utf8]{inputenc}
\usepackage[T1]{fontenc}
\usepackage[english]{babel}
\usepackage[protrusion=true,expansion=true]{microtype} % better typography

% Math
\usepackage{amsmath,amssymb,amsfonts,amsthm}
\usepackage{mathabx}

\newtheorem{thm}{Theorem}[section]
\newtheorem{defn}{Definition}[section]

\newcommand{\bbN}{\mathbb N} %the natural numbers
\newcommand{\bbZ}{\mathbb Z} %the integers
\newcommand{\bbQ}{\mathbb Q} %the rational numbers
\newcommand{\bbR}{\mathbb R} %the real numbers
\newcommand{\bbC}{\mathbb C} %the complex numbers

% Graphic stuff
\usepackage[pdftex]{graphicx}
\usepackage[usenames,dvipsnames,table]{xcolor}
\definecolor{shade}{RGB}{235,235,235}
\usepackage[pdftex,colorlinks=true]{hyperref}
\hypersetup
{
    bookmarksnumbered,
    linkcolor=RoyalBlue,
    anchorcolor=RoyalBlue,
    citecolor=RoyalBlue,
    urlcolor=RoyalBlue,
    pdfstartview={FitV},
    pdfdisplaydoctitle
}

% Tables
\usepackage{booktabs}
\usepackage[hang,small,bf]{caption}

% Debug, etc.
\usepackage{todonotes}

% Computer science stuff
\usepackage{clrscode3e}
\usepackage{verbatim}
\newcommand{\mono}[1]{{\ttfamily#1}}
\usepackage{listings}
\lstset
{
    tabsize=2,
    numbers=left,
    breaklines=true,
    backgroundcolor=\color{shade},
    framexleftmargin=0.05in,
    basicstyle=\ttfamily\small,
    numberstyle=\tiny,
    keywordstyle=\color{RoyalBlue},
    stringstyle=\color{Maroon},
    commentstyle=\color{ForestGreen},
    language=Matlab
}



\usepackage{subfig}
\newcommand{\subfigureautorefname}{\figureautorefname}


\title{Principles of Computer Systems Design -- Assignment 3}
\date{\today}
\author{Daniel Egeberg \and Søren Dahlgaard}

\begin{document}

\maketitle

\section{Exercises}

\subsection*{Question 1}

\subsection*{Question 2}

The Exokernel is an example of the End-to-end argument, in the way, that
there is no underlying abstractions that can slow the application down.
For instance, if you want to access a file, you can access the blocks directly,
and don't have to create file descriptors and check for access rights. This
is what most operating systems do in order to ensure security, which is more
important in the general case.

Encrypted data transmission has a three-fold end-to-end argument as
described in the course literature. The transmission system must be trusted
to manage the encryption keys, the authenticity of the data must still be
checked at high-level after transferred, and we note, that the data is
vulnerable while being transferred from the application to the transmission
system. This all talks in favor of a design like TLS/SSL, where the encryption
is a protocol on top of the transport layer. If we want to ensure, that data
cannot be accidentally or intentionally be sent un-encrypted, we must use
encryption in the transmission system, so as in most cases, it all depends
on the application scenario.

\subsection*{Question 3}

The daisy chain connects all buildings if both links work. This is easily
described as:
\begin{align*}
    \text{Pr}\{\text{daisy chain good}\} &= (1-p)^2 \\
                                         &= 1 - 2p + p^2
\end{align*}

The fully connected one works as long as at most one link is down. This can
be described as:
\begin{align*}
    \text{Pr}\{\text{fully connected good}\} &= (1-p)^3 - 3p(1-p)^2 \\
                                            &= 1 - 3p^2 + 2p^3
\end{align*}

The fully connected version provides better reliability since
$3\cdot (10^{-4})^2 < 2\cdot 10^{-6}$.


\section{Implementation}

\subsection*{Question 1}

\subsection*{Question 2}

\subsection*{Question 3}

\subsection*{Question 4}

%\begin{thebibliography}{9}
%\end{thebibliography}


\end{document}
