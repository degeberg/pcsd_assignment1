\documentclass[a4paper,final]{article}

\usepackage{a4wide}

% Language and encoding
\usepackage[utf8]{inputenc}
\usepackage[T1]{fontenc}
\usepackage[english]{babel}
\usepackage[protrusion=true,expansion=true]{microtype} % better typography

% Math
\usepackage{amsmath,amssymb,amsfonts,amsthm}
\usepackage{mathabx}

\newtheorem{thm}{Theorem}[section]
\newtheorem{defn}{Definition}[section]

\newcommand{\bbN}{\mathbb N} %the natural numbers
\newcommand{\bbZ}{\mathbb Z} %the integers
\newcommand{\bbQ}{\mathbb Q} %the rational numbers
\newcommand{\bbR}{\mathbb R} %the real numbers
\newcommand{\bbC}{\mathbb C} %the complex numbers

% Graphic stuff
\usepackage[pdftex]{graphicx}
\usepackage[usenames,dvipsnames,table]{xcolor}
\definecolor{shade}{RGB}{235,235,235}
\usepackage[pdftex,colorlinks=true]{hyperref}
\hypersetup
{
    bookmarksnumbered,
    linkcolor=RoyalBlue,
    anchorcolor=RoyalBlue,
    citecolor=RoyalBlue,
    urlcolor=RoyalBlue,
    pdfstartview={FitV},
    pdfdisplaydoctitle
}

% Tables
\usepackage{booktabs}
\usepackage[hang,small,bf]{caption}

% Debug, etc.
\usepackage{todonotes}

% Computer science stuff
\usepackage{clrscode3e}
\usepackage{verbatim}
\newcommand{\mono}[1]{{\ttfamily#1}}
\usepackage{listings}
\lstset
{
    tabsize=2,
    numbers=left,
    breaklines=true,
    backgroundcolor=\color{shade},
    framexleftmargin=0.05in,
    basicstyle=\ttfamily\small,
    numberstyle=\tiny,
    keywordstyle=\color{RoyalBlue},
    stringstyle=\color{Maroon},
    commentstyle=\color{ForestGreen},
    language=Matlab
}



\usepackage{subfig}
\newcommand{\subfigureautorefname}{\figureautorefname}


\title{Principles of Computer Systems Design -- Assignment 3}
\date{\today}
\author{Daniel Egeberg \and Søren Dahlgaard}

\begin{document}

\maketitle

\section{Exercises}

\subsection*{Question 1}

\subsection*{Question 2}

The Exokernel is an example of the End-to-end argument, in the way, that
there is no underlying abstractions that can slow the application down.
For instance, if you want to access a file, you can access the blocks directly,
and don't have to create file descriptors and check for access rights. This
is what most operating systems do in order to ensure security, which is more
important in the general case.

Encrypted data transmission has a three-fold end-to-end argument as
described in the course literature. The transmission system must be trusted
to manage the encryption keys, the authenticity of the data must still be
checked at high-level after transferred, and we note, that the data is
vulnerable while being transferred from the application to the transmission
system. This all talks in favor of a design like TLS/SSL, where the encryption
is a protocol on top of the transport layer. If we want to ensure, that data
cannot be accidentally or intentionally be sent un-encrypted, we must use
encryption in the transmission system, so as in most cases, it all depends
on the application scenario.

\subsection*{Question 3}

The daisy chain connects all buildings if both links work. This is easily
described as:
\begin{align*}
    \text{Pr}\{\text{daisy chain good}\} &= (1-p)^2 \\
                                         &= 1 - 2p + p^2
\end{align*}

The fully connected one works as long as at most one link is down. This can
be described as:
\begin{align*}
    \text{Pr}\{\text{fully connected good}\} &= (1-p)^3 - 3p(1-p)^2 \\
                                            &= 1 - 3p^2 + 2p^3
\end{align*}

The fully connected version provides better reliability since
$3\cdot (10^{-4})^2 < 2\cdot 10^{-6}$.


\section{Implementation}

\subsection*{Question 1}
The \mono{KeyValueBaseMasterImpl} has been given a \mono{replicatorImpl}, to
make sure all writes are replicated to the slaves. This invocation of the
replicator has been added to all modifyin operations (update, insert, delete,
bulkput and init). Note, that the init operation is easily done with the
replicator. The \mono{config} operation adds all the slaves to the replicator
by simply creating the services with the given url.

Also, we have create \mono{KeyValueBaseReplicaImpl} as a common base class
for the master and slave. The only change is, that it also stores the last
LSN, which is returned in all requests.

The \mono{ReplicatorImpl} keeps a list of all active slaves, and removes a
slave if it is unreachable.

\subsection*{Question 2}
The slave uses the \mono{KeyValueBaseReplicaImpl} as a base class, so it is
able to handle LSNs automatically. The only change to the slave is thus, that
it handles the logApply operation. This is just forwarded to the
\mono{logRecord}.

\subsection*{Question 3}
The majority of the proxy deals with transforming the data coming from the
slave/master services into data that can be returned to the sender. This
also includes exceptions, which have to be rethrown.

Each operation is performed by picking at random a master/slave (or just the
master if it's a modifying operation) and forwarding the operation to this
service. The proxy thus keeps a list of all available slaves and the master.

When it receives a LSN it compares the LSN to the one it keeps itself, and if
the received LSN is newer, the state of the proxy is updated. If it is older,
the operation is rerun on a new random slave/master until we receive an
up-to-date LSN. This can potentially cause a lot of false-negatives, which
will slow the performance, but it makes the state a lot easier to maintain
(only one LSN needs to be stored).

\subsection*{Question 4}

%\begin{thebibliography}{9}
%\end{thebibliography}


\end{document}
